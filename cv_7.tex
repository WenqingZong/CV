%%%%%%%%%%%%%%%%%%%%%%%%%%%%%%%%%%%%%%%%%
% "ModernCV" CV and Cover Letter
% LaTeX Template
% Version 1.3 (29/10/16)
%
% This template has been downloaded from:
% http://www.LaTeXTemplates.com
%
% Original author:
% Xavier Danaux (xdanaux@gmail.com) with modifications by:
% Vel (vel@latextemplates.com)
%
% License:
% CC BY-NC-SA 3.0 (http://creativecommons.org/licenses/by-nc-sa/3.0/)
%
% Important note:
% This template requires the moderncv.cls and .sty files to be in the same 
% directory as this .tex file. These files provide the resume style and themes 
% used for structuring the document.
%
%%%%%%%%%%%%%%%%%%%%%%%%%%%%%%%%%%%%%%%%%

%----------------------------------------------------------------------------------------
%	PACKAGES AND OTHER DOCUMENT CONFIGURATIONS
%----------------------------------------------------------------------------------------

\documentclass[11pt,a4paper,sans]{moderncv} % Font sizes: 10, 11, or 12; paper sizes: a4paper, letterpaper, a5paper, legalpaper, executivepaper or landscape; font families: sans or roman

\moderncvstyle{classic} % CV theme - options include: 'casual' (default), 'classic', 'oldstyle' and 'banking'
\moderncvcolor{blue} % CV color - options include: 'blue' (default), 'orange', 'green', 'red', 'purple', 'grey' and 'black'
\usepackage{multicol}
\usepackage{lipsum} % Used for inserting dummy 'Lorem ipsum' text into the template
\usepackage[hyperref]{}
\usepackage[scale=0.9]{geometry} % Reduce document margins
%\setlength{\hintscolumnwidth}{3cm} % Uncomment to change the width of the dates column
%\setlength{\makecvtitlenamewidth}{10cm} % For the 'classic' style, uncomment to adjust the width of the space allocated to your name

%----------------------------------------------------------------------------------------
%	NAME AND CONTACT INFORMATION SECTION
%----------------------------------------------------------------------------------------

\firstname{Wenqing} % Your first name
\familyname{Zong} % Your last name

% All information in this block is optional, comment out any lines you don't need
%\title{Curriculum Vitae}
% \address{116 RKB, IIT Roorkee}{Uttarakhand, India 247667}
\mobile{(+44) 07713 918152}
%\phone{(000) 111 1112}
%\fax{(000) 111 1113}
\email{wenqing.zong98@gmail.com}
% \email{asinha@mt.iitr.ac.in}
\homepage{zongwenqing.com}{zongwenqing.com} % The first argument is the url for the clickable link, the second argument is the url displayed in the template - this allows special characters to be displayed such as the tilde in this example
\extrainfo{\href{https://github.com/WenqingZong}{Github: WenqingZong}}
%\photo[70pt][0.4pt]{pictures/picture} % The first bracket is the picture height, the second is the thickness of the frame around the picture (0pt for no frame)
%\quote{"A witty and playful quotation" - John Smith}

%----------------------------------------------------------------------------------------

\begin{document}

%----------------------------------------------------------------------------------------
%	COVER LETTER
%----------------------------------------------------------------------------------------

% To remove the cover letter, comment out this entire block

%\recipient{HR Department}{Corporation\\123 Pleasant Lane\\12345 City, State} % Letter recipient
%\date{\today} % Letter date
%\opening{Dear Sir or Madam,} % Opening greeting
%\closing{Sincerely yours,} % Closing phrase
%\enclosure[Attached]{curriculum vit\ae{}} % List of enclosed documents
%
%\makelettertitle % Print letter title
%
%\lipsum[1-2] % Dummy text
%\lipsum[4] % Dummy text
%
%\makeletterclosing % Print letter signature
%
%\newpage

%----------------------------------------------------------------------------------------
%	CURRICULUM VITAE
%----------------------------------------------------------------------------------------

\makecvtitle % Print the CV title

%----------------------------------------------------------------------------------------
%	EDUCATION SECTION
%----------------------------------------------------------------------------------------

\section{Education}

\cventry{2021 - 2022}{Imperial College London}{\newline MSc.Advanced Computing}{}{}{Final Grade 70+}  % Arguments not required can be left empty

\cventry{2018 - 2021}{The University of Manchester}{\newline BSc.Artificial Intelligence}{}{}{Top 10\% Graduate}  % 
\cventry{2017 - 2018}{INTO Manchester}{\newline Foundation Year}{}{}{Top 10\% Graduate with A*A*A* in Maths, Further Maths and Physics}  %
%\section{%Masters Thesis}

%\cvitem{Title}{\emph{Money Is The Root Of All Evil -- Or Is It?}}
%\cvitem{Supervisors}{Professor James Smith \& Associate Professor Jane Smith}
%\cvitem{Description}{This thesis explored the idea that money has been the cause of untold anguish and suffering in the world. I found that it has, in fact, not.}

%----------------------------------------------------------------------------------------
%	WORK EXPERIENCE SECTION
%----------------------------------------------------------------------------------------

\section{Experience}

%\subsection{%Vocational}
\cventry{Sep 2023 - Present}{Software Engineer - Full Time}{Emotech}{London, UK}{Rust + Python}{
\begin{itemize}
\item Served as a key engineer on the ASR (Automatic Speech Recognition) team.
\item Responsible for backend API development, transforming AI models trained by the research team into market-ready products, employing innovative libraries and algorithms to minimize API response times.
\item Responsible for developing testing and visualization tools to monitor model performance.
\item Responsible for Azure SaaS integration to market our products.
\item Contributed to company's open source Silero VAD (Voice Activity Detection) library.
\item The product earned high praise from both colleagues and clients.
\end{itemize}}


\cventry{Jan 2023 - Aug 2023}{Software Engineer - Full Time}{Codethink}{Manchester, UK}{C}{
\begin{itemize}
\item Developed and contributed to Free and Open-Source Software to automate quality assurance testing.
\item Quality Assurance Deamon: Provides remote interaction with a device in place of having to physically interact with it. It's a remote control for test rigs. \href{https://www.codethink.co.uk/articles/2023/qad-for-hardware-testing/}{[Link to Blog Post]}
\item Testing in a Box: Integrates GitLab server/runner, OpenQA webUI/worker, and Q.A.D. into one box, making it an all-in-one solution for fully automated hardware testing. On-going project.
\end{itemize}}

\cventry{Jun 2021 - Aug 2021}{Machine Learning Engineer - Internship}{AgCIM Research Centre}{Guangzhou, China}{Python}{
\begin{itemize}
\item Utilized Pytorch to develop an image-based rural area hazard detection system with core functionalities such as object segmentation and road category classification.
\item Improved the accuracy of the road width calculation module in City Information Modeling (CIM) by incorporating the MegaDepth network.
\end{itemize}}

\cventry{Jun 2022 - Aug 2022}{Network Support Engineer - Part Time}{Sobey}{London}{}{
\begin{itemize}
\item Monitored the status of over 200 server clusters and ensure the proper functioning of the database.
\item Regularly performs maintenance on PCs for non-technical colleagues.
\end{itemize}}

\section{Skills}

\cvitem{Languages}{Proficient in Python, Rust, Java, C, familiar with JavaScript and C++}
\cvitem{Frameworks}{Tokio, Ndarray, PyTorch, PyTest, Flask, Spring, OpenGL, JUnit}
\cvitem{Utilities}{WebAssembly, Linux, Docker, Ansible, Git, Markdown, LaTeX, CI/CD, Nginx, AWS, Azure}
\cvitem{Communication}{English(fluent), Chinese(mother language)}

\section{Prizes}

\cvitem{Oct 2024}{Internet Computer's 5000 USDC prize, awarded at Encode London 2024 Hackathon}

\section{Projects}

\cventry{Oct 2024 - Present}{Python Debug Library}{}{\href{https://pypi.org/project/crab_dbg/}{[PyPI]}}{Python}{
\begin{itemize}
\item Easily inspects values of variables and expressions in a human-readable way.
\item Supports primitive types, user-defined classes,  nested objects and recursive objects.
\item Can be used as a drop-in replacement of python's print() function.
\end{itemize}}

\cventry{Aug 2024 - Present}{Silero Voice Activity Detection Library}{}{\href{https://github.com/emotechlab/silero-rs}{[GitHub]}}{Rust}{
\begin{itemize}
\item Pure Rust implementation of Silero VAD model and algorithm. Support running on any hardware.
\item Easy-to-use batch and streaming interface, plus all the utilities you'd expect to see in an audio project.
\item In progress: Async interface. 
\end{itemize}}

\cventry{Jun 2023 - Jul 2023}{Rust Octree Library}{}{\href{https://github.com/WenqingZong/Octree}{[Github]}}{Rust}{
\begin{itemize}
\item A highly optimised Octree implementation.
\item Capable of tracking dynamic objects in the environment.
\item Easy to integrate into existing codebase.
\end{itemize}}

\cventry{Mar 2023 - May 2023}{Brainf*ck Interpreter in Rust}{}{\href{https://github.com/WenqingZong/BrainfuckInterpreter}{[Github]}}{Rust}{
\begin{itemize}
\item Developed a highly optimized interpreter for the Brainf*ck language using Rust.
\item Implemented a modern and user-friendly command-line interface.
\item Included extensive debugging messages for static checking and runtime errors.
\item Achieved high test coverage and fully documented the project.
\end{itemize}}

\cventry{May 2022 - Sep 2022}{Unsupervised Domain Adaptation on Medical Images}{\textit{Dr.  Matthew Williams, Imperial College London}}{\href{https://github.com/WenqingZong/Momentum-Prototype-UDA}{[Github]}}{PyTorch}{
\begin{itemize}
\item Devised a novel method for addressing the domain shift problem, enabling a model trained on one dataset to adapt and fit to another dataset without significant loss in performance.
\item The proposed novel method offers two key benefits:
\\
1. Source-Free: Model adaptation does not require the source dataset, which enhances cross-institutional collaboration efficiency and addresses data privacy.
\\
2. Supports Various Network Backbones: The novel method is compatible with all neural network architectures, without any special requirements.
\item Demonstrated the efficacy of the proposed method on BraTS2021 dataset, achieving comparable performance with the state-of-the-art approach.
\end{itemize}}

\cventry{Jan 2022 - Mar 2022}{Robot Learning and Control in Maze Environment}{Self-motivated}{PyTorch}{}{
\begin{itemize}
\item Implemented several algorithms to teach a robot how to solve a maze.
\item Traditional algorithm: Cross Entropy Method. Continuously adjusted the covariance matrix to make the action distribution approach the known optimal solution.
\item Machine Learning: Trained a model to learn the non-linear environment and later used in Model Predictive Control algorithm.
\item Behavioural Cloning. Trained a model to mimic how humans navigate in the maze. Implemented the DAgger algorithm to improve the model's performance while reducing the amount of data needed.
\end{itemize}}

\cventry{Oct 2020 - Apr 2021}{Procedural Terrain Generation for Video Game Development}{\textit{Dr Ke Chen, The University of Manchester}}{PyTorch, C\#, Unity}{}{
\begin{itemize}
\item Utilized Perlin noise to procedurally generate terrains for modern RPG games and simulated hydraulic erosion process to enhance playability.
\item Employed Spatial GAN model to generate realistic terrain for a flight simulation game.
\item Completed as a First Class Final Year project for my undergraduate degree.
\end{itemize}}
 
\cventry{Oct 2020 - Dec 2020}{MCTS Board Game AI}{Team}{\href{https://github.com/WenqingZong/Kalah_AI}{[Github]}}{Java}{
\begin{itemize}
\item Collaborated with a team of four to develop an AI bot to play a board game, Kalah.
\item Implemented a bot based on Monte Carlo Tree Search with some improvements such as Early Payout Termination and MCTS-Minimax hybrid.
\item Our bot beats 37 bots submitted by other teams (51 in total) in a tournament.
\end{itemize}}

%----------------------------------------------------------------------------------------
%	INTERESTS SECTION
%----------------------------------------------------------------------------------------

%\section{Interests}

%\renewcommand{\listitemsymbol}{-~} % Changes the symbol used for lists

%\cvlistdoubleitem{Cycling}{Hiking}
%\cvlistdoubleitem{Sketching}{Gaming}
%\cvlistitem{Quizzing}
%----------------------------------------------------------------------------------------

% \section{References}

% \begin{multicols}{2}
% \cventry{}{K.S Suresh}{\newline Assistant Professor}{\newline Metallurgical and Materials Engg., IIT Roorkee}{\newline suresfmt@iitr.ac.in}{}
% \columnbreak
% \cventry{}{Anu Chandra}{\newline CEO}{\newline Ryelore AI}{\newline anu@ryelore.com}{ }
% \end{multicols}
%\cventry{}{Arpit Gupta}{\newline VP Engineering}{\newline Antriex IT Services}{\newline arpit.gupta@antmex.com}{}

%\cventry{}{B.S.S Daniel}{\newline Professor}{\newline Metallurgical and Materials Engg., IIT Roorkee}{\newline s4danfmt@iitr.ac.in}{ }
\end{document}
