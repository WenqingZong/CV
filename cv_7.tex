%%%%%%%%%%%%%%%%%%%%%%%%%%%%%%%%%%%%%%%%%
% "ModernCV" CV and Cover Letter
% LaTeX Template
% Version 1.3 (29/10/16)
%
% This template has been downloaded from:
% http://www.LaTeXTemplates.com
%
% Original author:
% Xavier Danaux (xdanaux@gmail.com) with modifications by:
% Vel (vel@latextemplates.com)
%
% License:
% CC BY-NC-SA 3.0 (http://creativecommons.org/licenses/by-nc-sa/3.0/)
%
% Important note:
% This template requires the moderncv.cls and .sty files to be in the same 
% directory as this .tex file. These files provide the resume style and themes 
% used for structuring the document.
%
%%%%%%%%%%%%%%%%%%%%%%%%%%%%%%%%%%%%%%%%%

%----------------------------------------------------------------------------------------
%	PACKAGES AND OTHER DOCUMENT CONFIGURATIONS
%----------------------------------------------------------------------------------------

\documentclass[11pt,a4paper,sans]{moderncv} % Font sizes: 10, 11, or 12; paper sizes: a4paper, letterpaper, a5paper, legalpaper, executivepaper or landscape; font families: sans or roman

\moderncvstyle{classic} % CV theme - options include: 'casual' (default), 'classic', 'oldstyle' and 'banking'
\moderncvcolor{blue} % CV color - options include: 'blue' (default), 'orange', 'green', 'red', 'purple', 'grey' and 'black'
\usepackage{multicol}
\usepackage{lipsum} % Used for inserting dummy 'Lorem ipsum' text into the template
\usepackage[hyperref]{}
\usepackage[scale=0.9]{geometry} % Reduce document margins
%\setlength{\hintscolumnwidth}{3cm} % Uncomment to change the width of the dates column
%\setlength{\makecvtitlenamewidth}{10cm} % For the 'classic' style, uncomment to adjust the width of the space allocated to your name

%----------------------------------------------------------------------------------------
%	NAME AND CONTACT INFORMATION SECTION
%----------------------------------------------------------------------------------------

\firstname{Wenqing} % Your first name
\familyname{Zong} % Your last name

% All information in this block is optional, comment out any lines you don't need
%\title{Curriculum Vitae}
% \address{116 RKB, IIT Roorkee}{Uttarakhand, India 247667}
\mobile{(+44) 07713 918152}
%\phone{(000) 111 1112}
%\fax{(000) 111 1113}
\email{wenqing.zong98@gmail.com}
% \email{asinha@mt.iitr.ac.in}
% \homepage{sinashish.github.io}{www.sinashish.github.io} % The first argument is the url for the clickable link, the second argument is the url displayed in the template - this allows special characters to be displayed such as the tilde in this example
\extrainfo{\href{https://github.com/WenqingZong}{Github: WenqingZong}}
%\photo[70pt][0.4pt]{pictures/picture} % The first bracket is the picture height, the second is the thickness of the frame around the picture (0pt for no frame)
%\quote{"A witty and playful quotation" - John Smith}

%----------------------------------------------------------------------------------------

\begin{document}

%----------------------------------------------------------------------------------------
%	COVER LETTER
%----------------------------------------------------------------------------------------

% To remove the cover letter, comment out this entire block

%\recipient{HR Department}{Corporation\\123 Pleasant Lane\\12345 City, State} % Letter recipient
%\date{\today} % Letter date
%\opening{Dear Sir or Madam,} % Opening greeting
%\closing{Sincerely yours,} % Closing phrase
%\enclosure[Attached]{curriculum vit\ae{}} % List of enclosed documents
%
%\makelettertitle % Print letter title
%
%\lipsum[1-2] % Dummy text
%\lipsum[4] % Dummy text
%
%\makeletterclosing % Print letter signature
%
%\newpage

%----------------------------------------------------------------------------------------
%	CURRICULUM VITAE
%----------------------------------------------------------------------------------------

\makecvtitle % Print the CV title

%----------------------------------------------------------------------------------------
%	EDUCATION SECTION
%----------------------------------------------------------------------------------------

\section{Education}

\cventry{2021 - Present}{Imperial College London}{\newline MSc.Advanced Computer science}{}{}{First Class Expected}  % Arguments not required can be left empty

\cventry{2018 - 2021}{The University of Manchester}{\newline BSc.Artificial Intelligence}{}{}{Top 10\% Graduate}  % 
\cventry{2017 - 2018}{INTO Manchester}{\newline Foundation Year}{}{}{Top 10\% Graduate}  %
%\section{%Masters Thesis}

%\cvitem{Title}{\emph{Money Is The Root Of All Evil -- Or Is It?}}
%\cvitem{Supervisors}{Professor James Smith \& Associate Professor Jane Smith}
%\cvitem{Description}{This thesis explored the idea that money has been the cause of untold anguish and suffering in the world. I found that it has, in fact, not.}

%----------------------------------------------------------------------------------------
%	WORK EXPERIENCE SECTION
%----------------------------------------------------------------------------------------

\section{Experience}

%\subsection{%Vocational}

\cventry{Jun 2021 - Aug 2021}{Research Internship}{AgCIM Research Centre}{Guangzhou, China}{}{
\begin{itemize}
\item Completed image-based rural area hazard detection system using Pytorch. Core functions include object segmentation and road category classification.
\item Used MegaDepth network to improve the accuracy of road width calculation module in City Information Modeling (CIM).
\item Work highly recognized by my colleagues.
\end{itemize}}


\section{Skills}

\cvitem{Languages}{Python, familiar with Java and other C-like}
\cvitem{Frameworks}{PyTorch, Numpy, familiar with Tensorflow, Spring, OpenGL}
% \cvitem{WebD}{HTML/CSS, JavaScript, Jekyll}
\cvitem{Utilities}{Anaconda, IDEA, Jupyter Notebook, Git, Markdown, LaTeX}
\cvitem{Communication}{English(fluent), Chinese(mother language)}


\section{Projects}

\cventry{May 2022 - Present}{Transfer Learning on Brain Metastasis Detection}{\textit{Dr.  Matthew Williams, Imperial College London}}{}{}{
\begin{itemize}
\item Apply transfer learning on CT scans to detect Brain Metastasis.
\item On going project.
\end{itemize}}

\cventry{Oct 2020 - Apr 2021}{Procedural Terrain Generation}{\textit{Dr Ke Chen, The University of Manchester}}{}{}{
\begin{itemize}
\item Used Perlin noise to procedurally generate terrains for modern RPG games and simulated hydraulic erosion process to increase playability. 
\item Used Spatial GAN model to generate realistic terrain for flight simulation type game.
\item First Class Final year project (Undergraduate).
\end{itemize}}

\cventry{Oct 2020 - Dec 2020}{N Body Movement Simulation}{Self-motivated}{\href{https://github.com/WenqingZong/N-Body-Simulator}{[Code]}}{}{
\begin{itemize}
\item Built a particle system to simulate the N-Body movement problem in OpenGL.
\item Used PyQt framework to provide a powerful GUI where users can adjust all parameters of each particle.
\end{itemize}}
 
\cventry{Oct 2020 - Dec 2020}{MCTS Board Game AI}{Team}{\href{https://github.com/WenqingZong/Kalah_AI}{[Code]}}{}{
\begin{itemize}
\item Participated in a team of 4 to develop an AI bot to play a board game, Kalah. 
\item Our bot was based on Monte Carlo Tree Search with some improvements such as Early Payout Termination, MCTS-Minimax hybrid. 
\item The bot beats 37 bots submitted by other teams (51 in total) in a tournament.
\end{itemize}}
 
\cventry{Jan 2020 - May 2020}{EventLite Website}{Team}{}{}{
\begin{itemize}
\item Lead a team of 6 people to develop a website, EventLite, in Spring framework.
\item Set up and maintained the website database.
\item Code is maintained in a high standard with unit tests and security tests of each function we implemented.
\end{itemize}}

\cventry{Nov 2019}{Face Recognition}{Self-motivated}{}{}{
\begin{itemize}
\item Trained a model by minimising the $L2$-regularised sum of squares loss using the normal equations. 
\item Achieved 92.5\% accuracy when classifying 40 people. 
\end{itemize}}

\cventry{Oct 2019 - Jan 2020}{Stendhal Game}{Team}{}{}{
\begin{itemize}
\item Worked in a group of 7 to maintain an open source game in GitLab. 
\item Fixed bugs raised by players and added JUnit tests for them. Introduced new features based on the original game and refactored some hard-to-read legacy code.
\end{itemize}}

\cventry{Jan 2019}{Yes/No Voice Recognition System}{Self-motivated}{}{}{
\begin{itemize}
\item Built a voice recognition system to distinguish "yes" and "no" said by different people with different accent. 
\item Originally this was a naive Bayes classifier, optimised with Markov chain and hidden Markov Model later on. Achieved 89\% testing accuracy.
\end{itemize}}

\cventry{Jan 2019}{Robot Localization System}{Self-motivated}{}{}{
\begin{itemize}
\item Developed an automatic robot positioning system, which is based on Bayes probability theory. 
\item Achieve high positioning accuracy even when sensor readings maybe inaccurate, thereby reducing the probability of collision between the robot and obstacles.
\end{itemize}}


\section{Extra Curriculars}

\cventry{Sep 2019 - Jun 2020}{PASS Leader}{The University of Manchester}{}{}{ Helped several first year students in Peer Assisted Study Session, offering academic support and developed interpersonal skills.
}

\cventry{Sep 2018 - Jun 2019}{Student Representative}{The University of Manchester}{}{}{
Acted on behalf of students to raise issues we concerned and suggestions to department staff, in order to make a positive difference not only on courses but also the whole CS department community.
}

%----------------------------------------------------------------------------------------
%	INTERESTS SECTION
%----------------------------------------------------------------------------------------

%\section{Interests}

%\renewcommand{\listitemsymbol}{-~} % Changes the symbol used for lists

%\cvlistdoubleitem{Cycling}{Hiking}
%\cvlistdoubleitem{Sketching}{Gaming}
%\cvlistitem{Quizzing}
%----------------------------------------------------------------------------------------

% \section{References}

% \begin{multicols}{2}
% \cventry{}{K.S Suresh}{\newline Assistant Professor}{\newline Metallurgical and Materials Engg., IIT Roorkee}{\newline suresfmt@iitr.ac.in}{}
% \columnbreak
% \cventry{}{Anu Chandra}{\newline CEO}{\newline Ryelore AI}{\newline anu@ryelore.com}{ }
% \end{multicols}
%\cventry{}{Arpit Gupta}{\newline VP Engineering}{\newline Antriex IT Services}{\newline arpit.gupta@antmex.com}{}

%\cventry{}{B.S.S Daniel}{\newline Professor}{\newline Metallurgical and Materials Engg., IIT Roorkee}{\newline s4danfmt@iitr.ac.in}{ }
\end{document}